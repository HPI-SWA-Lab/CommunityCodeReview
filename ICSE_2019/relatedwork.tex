% !TeX spellcheck = en_us
%!TEX root = main.tex

\section{Related Work}
We build on \cite{Duerschmid2017CCR}


Many projects working with feature branches perform code reviews during pull requests before merging the changes into the main branch~\cite{driessen2010successful, calefato2015PLE, yu2015pullrequests, tsay2014contributionGithub, gousios2014PullBasedSD, rahman2014pullrequests, tsay2014ContributionDiscussion}. 
%
Tools like CodeFlow~\cite{bird2015CodeReviewPlatform}, Mondrian~\cite{kennedy2006Mondrian}, Gerrit~\cite{google2016gerrit},  Phabricator~\cite{tsotsis2011Phabricator}, ClusterChanges~\cite{barnett2015helpingdevelopers} and the Eclipse plug-in EGerrit\footnote{\url{https://www.eclipse.org/egerrit/}} support these change-based code reviews.
%
In contrast, our concept gives feedback on the current state of code. 
%
The reviews supported by these tools are made using a push model, meaning that developers request reviews, while our tool uses a pull model, meaning that developers can comment on any code without request of the author.
%
Hence, our concept allows to comment on library code and framework code and therefore to give feedback to the developers of third party software. 
%
Furthermore, our feedback is continuous and therefore enables new developers to comment on old code. 
%
Similar to plug-ins like EGerrit, the proposed tool enables the developers to stay inside their IDE and avoid context switches.
%
This provides a more self-sustaining environment which supports the liveness of the development process. 
%
However, a dedicated review process before merging changes into the main branch lets the developers focus on issues raised up by the specific changes. 
%
Therefore, we propose to use our concept in conjunction with pull requests or other change-based reviews and not instead of it.
%

%
Discussions and questions about the source code traditionally are done using mailing lists~\cite{vasilescu2014QA}.
%
The recent emergence of questions \& answers (Q\&A) sites introduces gamification of archiving reputation for answering questions~\cite{vasilescu2014QA}.
%
But like pull requests, they usually are not done inside the IDE but on a separate StackExchange network like StackOverflow.