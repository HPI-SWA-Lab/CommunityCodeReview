% !TeX spellcheck = en_us
%!TEX root = main.tex

\section{Introduction}
Conducting code reviews is an important practice for both professional companies as well as open source projects~\cite{balachandran2013PeerCodeReviews, bird2015CodeReviewPlatform, rigby2013PeerCodeReviews, czerwonka2015codereviews, rigby2014PeerReviewOSS, feitelson2013development}.
%
Developers' reasons for code reviews are improving code quality for maintainability, transferring knowledge, finding defects, discussing alternative solutions, and improving team awareness~\cite{rigby2013PeerCodeReviews, bacchelli2013expectations, bosu2017ContemporaryCodeReview}.
%
Recent studies confirmed, that review coverage and review participation have a significant impact on code quality and the correctness of software~\cite{mcintosh2014impact, mcintosh2016empirical, thongtanunam2015CodeReviews, shimagaki2016CRInSony}. 
%

%
But in practice, code reviews are done only once before the code is merged into the main branch~\cite{rigby2013PeerCodeReviews}. 
%
This process does not continuously give feedback on code quality---especially of legacy code---, does not support questions of new developers concerning existing code, and forces developers to leave their IDE for commenting on code.
%

%
Social networks, such as Facebook\footnote{\url{https://facebook.com}}, Twitter\footnote{\url{https://twitter.com}}, or Stack Overflow\footnote{\url{https://stackoverflow.com}}, enable interaction with any content by adding a comment or pushing an \enquote{I like} button.
% 
We transfer this metaphor to programming by offering developers to comment on any code and to like clean code snippets in the project using their IDE.  
%

%
Code reviews of large changes (20 files or more) are less useful~\cite{czerwonka2015codereviews}.
%
Therefore, the proposed tool tries to continuously give feedback just while reading a piece of code inside the IDE. 
%
This approach lowers the barrier to comment on source code and increases the amount of feedback.
%
%Furthermore, it can be enriched by applying gamification techniques to commenting on source code and finding bugs~\cite{lotufo2012towards}.
%